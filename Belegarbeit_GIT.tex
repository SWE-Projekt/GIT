\documentclass[a4paper]{article}
\usepackage{ngerman, amsmath, amssymb, bbm, array, tabularx}
\usepackage[T1]{fontenc}
\usepackage[utf8]{inputenc}
\setlength{\parskip}{1.5ex plus0.5ex minus0.2ex}
\setlength{\parindent}{0cm}
\usepackage[]{graphicx}


\begin{document}

\title{\textbf{{\Huge Belegarbeit GIT}}}
\author{Fabian Goerge \\  Sami Ede \\ Richard Peters \\ Jens Meise}
\date{18.11.12}
\maketitle
\newpage
\tableofcontents
\newpage
\section{\underline{Einführung}}
\subsection{GIT Projekt}

\par
\begingroup
\leftskip=1,1cm 
\noindent GIT ist ein Software zur Versionsverwaltung....
\par
\endgroup


	
\section{\underline{Modellierung}}

\subsection{Use Cases}

	Hier gehts los ....
		
\subsection{Domänenmodell, Glossar}

	Hier gehts los ....
	
\subsection{Verhaltensmodell (Zusatandsdiagramm)}

	Hier gehts los ....
	
\section{\underline{Gewählter Prozess / Erfahrung der Teamarbeit}}

\section{\underline{Zusammenfassung / Ausblick}}
\end{document}

